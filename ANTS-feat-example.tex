\Echapter{Using ANTs Registration with FEAT}{Kelly A. Sambrook}{kelly89@uw.edu}
\label{section:antsreg}

This is an example of using Advanced Normalization Tools (ANTs)\citep{ants}
registration with FSL Feat. ANTs provides a superior nonlinear
registration from the T1 to MNI brain compared to FSL's \texttt{fnirt}
command, so we use it here in conjunction with FSL's \texttt{epi-reg}
to replace the functional to standard space registrations created by
FEAT with those created by ANTs. 

The code for this example is in
\texttt{\$MAKEPIPELINES/ANTsreg/} in specific makefiles described
below. It is assumed that the fMRI data here have been pre-processed
(including motion correction) before running this pipeline. 

Note that to run this example, and indeed, to use ANTs with FSL, you
must have \href{http://stnava.github.io/ANTs/}{ANTs} and
\href{http://www.itksnap.org/pmwiki/pmwiki.php?n=Convert3D.Documentation}{ITK-SNAP
  Convert3D tools} installed. We assume that Convert3D tools
(specifically, \texttt{c3d_affine_tool}) are installed in your bin
directory and is available in your path. You can type:
\bashcmd{which c3d_affine_tool}

to determine where it is located if it is in your path.
ANTs, however, is another story; you may have a different
version than we do and it may be installed in a different location. So
please note that you will probably need to edit this example to
specify your location for ANTs.

\section{Group Level Makefile}
\begin{lstlisting}
	SUBJECTS = TS001 TS002 
	.PHONY:  all $(SUBJECTS)

	all:  $(SUBJECTS)

	$(SUBJECTS): 
		$(MAKE) --directory=$@ $(TARGET)
\end{lstlisting}

The code for this makefile can be found at \texttt{\$MAKEPIPELINES/ANTsreg/Makefile}. There are two subjects in this directory that require the same
subject-level processing. Of course, in a real study there would be
many more. This group level makefile is a driver that will
recursively call targets defined the subject-level makefiles. See
\nameref{sec:analysisdir} for more information on this type of
directory structure.

\section{Subject Level Makefile}
The subject-level makefile is in
\texttt{ANTsreg/Makefile.subject}. Each subject directory has a
symbolic link to this file that is named \texttt{Makefile}. In this
way, we can change it in the top-level directory but all subjects will
immediately see the changes.

\begin{lstlisting}
	cwd = $(shell pwd)
	subject=$(notdir $(cwd))
\end{lstlisting}
We use the directory name here to set a variable that contains the
name of the subject. This is handy when full pathnames must be
specified to programs. 

\begin{lstlisting}
	%*\lnote*export OMP_NUM_THREADS=1

	SHELL=/bin/bash
	ifeq "$(origin MAKEPIPELINES)" "undefined"
	MAKEPIPELINES=/project_space/makepipelines
	endif

	PROJECT_DIR=$(MAKEPIPELINES)/ANTsreg
	STANDARD_DIR=$(PROJECT_DIR)/Standard
	SUBJECT_DIR=$(PROJECT_DIR)/$(subject)

	%*\lnote*ANTSpath=/usr/local/ANTs-2.1.0-rc3/bin/

	%*\lnote*BLOCKS= TappingBlock1 TappingBlock2

	%*\lnote*FSLDIR=/usr/share/fsl/5.0/

	STD_BRAIN=$(FSLDIR)/data/standard/MNI152_T1_2mm_brain.nii.gz
	STD=$(FSLDIR)/data/standard/MNI152_T1_2mm.nii.gz
\end{lstlisting}

Here we define a number of variables that are used by other
makefiles. \\
\lnum{1} ANTs exploits multiple cores to speed up
execution. However, if we are running this pipeline across many
subjects using \maken{}, we disable this parallelism by setting the
variable \texttt{OMP_NUM_THREADS} to be one. \\
\lnum{2} This is where
ANTs is located on our systems. It is unlikely that ANTs is in the
same directory on your system, so you will have to edit this line.\\
\lnum{3} Similarly, the location of FSL may be different on your
system. Check the location of \texttt{FSLDIR} on your system and
uncomment it as shown in this example. Note that if FSL is in a
different location on your system, you will have to edit the Feat template files, located in
\texttt{ANTsreg/templates}. \\
\lnum{4} We set a variable
called \texttt{BLOCKS} to the names of the fMRI tapping task
runs. This saves some typing later.\\

\begin{lstlisting}
	antsreg: PrepSubject Feat

	include ../lib/makefiles/Prep.mk
	include ../lib/makefiles/Feat.mk

	clean:	 clean_prep clean_feat 
\end{lstlisting}
The main target here, \texttt{antsreg}, creates some basic
registrations (defined by the \texttt{PrepSubject} target in the
\texttt{Prep.mk} makefile) and then runs FEAT, creating ANTs
registrations in the FEAT directories (target \texttt{Feat} defined in
the \texttt{Feat.mk} makefile). These makefiles are stored in the
\texttt{lib} directory and are included by the subject-level makefile.

Finally, we define a \texttt{clean} target that depends upon the
\texttt{clean_prep} and \texttt{clean_feat} targets, defined in their
respective makefiles.

\section{Preparatory Registrations - \texttt{Prep.mk}}

For convenience we have separated the main work of performing
registrations with ANTs from the task of running Feat and applying the
ANTs registrations to the Feat output. The registrations are defined
in the Makefile \texttt{ANTsreg/lib/makefiles/Prep.mk}, described in
this section.

\begin{lstlisting}
	.PHONY: PrepSubject bet struct_registrations epi_registrations clean_prep
	PrepSubject: bet struct_registrations epi_registrations 

	%*\lnote*Drop1Sfx = $(basename $(1))
\end{lstlisting}
We define phony targets as usual, and the main target is the
\texttt{PrepSubject} target. \lnum{5} We also define a helpful little
function, \texttt{Drop1Sfx} that drops the suffix of its
argument. This is accomplished by using the \maken{} defined function
\texttt{basename}. 

\begin{lstlisting}
	%*\lnote*bet: MPRAGE/T1_brain.nii.gz $(patsubst %,Tapping/%_brain.nii.gz, $(BLOCKS))

	MPRAGE/T1_brain.nii.gz: MPRAGE/T1.nii.gz
		bet $^ $@ -R

	%*\lnote*Tapping/%_brain.nii.gz: Tapping/%.nii.gz
		bet $< $@ -F
\end{lstlisting}

We define a phony target, \texttt{bet}, to perform skull stripping of
the MPRAGE and functional images. \lnum{6} We use pattern substitution
on the \texttt{BLOCKS} variable to form the names of the
skull-stripped functional images. \lnum{7} An implicit rule is used to
create the skull-stripped functional images. This simplifies the
Makefile when there are multiple functional runs that all need to be
processed in the same way.

\begin{lstlisting}
	struct_registrations: xfm_dir/T1_to_mni_Warp.nii.gz

	xfm_dir/T1_to_mni_Warp.nii.gz: MPRAGE/T1_brain.nii.gz 
		mkdir -p xfm_dir ;\
		export ANTSPATH=$(ANTSpath) ;\
		$(ANTSpath)/antsIntroduction.sh -d 3 -i MPRAGE/T1_brain.nii.gz \
		-m 30x90x20 -o $(SUBJECT_DIR)/xfm_dir/T1_to_mni_ \
		-s CC -r $(STD_BRAIN) -t GR
\end{lstlisting}
The structural registration of the MPRAGE to the standard brain is
performed using ANTs, using the script
\texttt{antsIntroduction.sh}. The target file
\texttt{T1_to_mni_Warp.nii.gz} will be created by this script because
we have specified the output prefix using the \texttt{-o} flag.

\begin{lstlisting}
	epi_registrations: $(patsubst %,xfm_dir/%_to_T1.mat,$(BLOCKS)) 
	$(patsubst %,xfm_dir/%_to_mni_epireg_ants.nii.gz,$(BLOCKS))

	%*\lnote*Tapping/%_brain_vol0.nii.gz: Tapping/%_brain.nii.gz
		fslroi $< $@ 0 1

	xfm_dir/%_to_T1.mat: Tapping/%_brain_vol0.nii.gz MPRAGE/T1.nii.gz 
	MPRAGE/T1_brain.nii.gz
		mkdir -p xfm_dir ;\
		%*\lnote*epi_reg --epi=Tapping/$*_brain_vol0  \
		--t1=MPRAGE/T1.nii.gz --t1brain=MPRAGE/T1_brain.nii.gz \
		--out=$(call Drop1Sfx,$@)
\end{lstlisting}
We use the target \texttt{epi_registrations} to perform the
registrations involving the functional data. Our first step is to use
\texttt{epi_reg} from FSL to perform boundary-based registration of
the functional image to the MPRAGE. \lnum{8} We take the
first volume of each skull-stripped functional run to use as the input
volume for \texttt{epi_reg}. This neat trick (thanks Katie Askren!)
allows \texttt{epi_reg} to run faster and use a lot less
memory. \lnum{9} Note that we get to use the \texttt{Drop1Sfx}
function that we defined earlier on the target to specify the output
path to \texttt{epi_reg}.


\begin{lstlisting}
	%*\lnote*xfm_dir/%_to_T1_ras.txt: MPRAGE/T1_brain.nii.gz Tapping/%_brain_vol0.nii.gz xfm_dir/%_to_T1.mat
		c3d_affine_tool -ref MPRAGE/T1_brain.nii.gz \
		-src Tapping/$*_brain_vol0 xfm_dir/$*_to_T1.mat -fsl2ras -oitk $@

	%*\lnote*xfm_dir/%_to_mni_epireg_ants.nii.gz: Tapping/%_brain_vol0.nii.gz $(STD_BRAIN) MPRAGE/T1_brain.nii.gz xfm_dir/%_to_T1_ras.txt xfm_dir/T1_to_mni_Warp.nii.gz
		export ANTSPATH=$(ANTSpath) ;\
		$(ANTSpath)/WarpImageMultiTransform 3 Tapping/$*_brain_vol0.nii.gz $@ \
		-R $(STD_BRAIN) xfm_dir/T1_to_mni_Warp.nii.gz \
		xfm_dir/T1_to_mni_Affine.txt xfm_dir/$*_to_T1_ras.txt
\end{lstlisting}
Now comes the moment where we combine the T1 to standard space
registration that we made with ANTs with the functional to T1
registration that we did with \texttt{epi_reg}. To do this you need to
use \texttt{WarpImageMultiTransform} from the ANTs
package. \lnum{10} We first convert the FSL style transformation matrices to ITK
format using \texttt{c3d_affine_tool}. \lnum{11} Then we use
\texttt{WarpImageMultiTransform} to combine this matrix with the T1 to
standard space warp and affine matrix. Voila! We have achieved
registration. 

\begin{lstlisting}
	clean_prep:
		rm -rf  */*_brain*.nii.gz */*vol0.nii.gz xfm_dir
\end{lstlisting}

Finally, we define a target \texttt{clean_prep} to remove all files
created by this makefile.


\section{Running Feat and Applying ANTs Registrations -
  \texttt{Feat.mk}}
The next step is to run first level Feats, apply the ANTs
registrations to the resulting output files, and then run a second
level Feat for each individual. This is done by the Makefile described
here, located in \texttt{ANTsreg/lib/makefiles/Feat.mk}.

\begin{lstlisting}
	TEMPLATES=$(PROJECT_DIR)/templates

	Tapping_FEAT1_TEMPLATE=$(PROJECT_DIR)/templates/Tapping_FL.fsf
	Tapping_FEAT2_TEMPLATE=$(PROJECT_DIR)/templates/Tapping_HL.fsf	
\end{lstlisting}

We run Feat on multiple blocks with multiple subjects by defining Feat
templates. We create these by setting up Feat runs for a single
subject using the graphical user interface and saving the
configuration file. Then we edit that file, replacing things like
subject ID and the run name with capitalized text strings that we can
find and replace in a makefile with their real values. Here we define
variables for these templates (first level and higher level).


\begin{lstlisting}
	.PHONY: Feat FirstLevelFeats FirstLevelReg SecondLevelFeats
	.SECONDARY: $(allupdatereg)

	%*\lnote*allcopes=$(wildcard Tapping/*.feat/stats/cope*.nii.gz)
	allvarcopes=$(wildcard Tapping/*.feat/stats/varcope*.nii.gz)
	%*\lnote*allupdatereg=$(subst stats,reg_standard/stats,$(allcopes)) 
	$(subst stats,reg_standard/stats,$(allvarcopes))
	allupdatetdof= $(subst stats/cope,reg_standard/stats/FEtdof_t,$(allcopes)) 

	Feat: FirstLevelFeats FirstLevelReg SecondLevelFeats
\end{lstlisting}
We define our phony targets as usual. 

Our goal is to transform the copes and varcopes into standard space
using the ANTs registration. \lnum{12} However, because there may be an
arbitrary number of copes, we use a wildcard here to obtain their
names. \lnum{13} We use pattern substitution on the names that we
obtained from wildcards to define the filenames for the updated
registrations that we need to create. Similarly, we also define the
DOF (degrees of freedom) files that would normally be created by Feat.


\begin{lstlisting}
	FirstLevelFeats: $(patsubst %,Tapping/%.feat/stats/cope4.nii.gz, $(BLOCKS))

	Tapping/%.feat/stats/cope4.nii.gz: $(patsubst %,Tapping/%_brain.nii.gz, TappingBlock1 TappingBlock2)
		rm -rf `echo $@ | awk -F "/" `{print $$1"/"$$2}'' ;\
		NBVOLS=`fslval $(word 1,$^) dim4' ;\
		RUN=$* ;\
		RUN_NUM=`echo $${RUN} | sed -e `s/TappingBlock//g'' ;\
		sed -e `s|SUBDIR|$(SUBJECT_DIR)|g' -e "s/NB/$${NBVOLS}/g" \
		-e "s/RUN/$${RUN_NUM}/g" -e "s/SUBJECT/$(subject)/g" \
		$(Tapping_FEAT1_TEMPLATE) > Tapping/$*.fsf ;\
		feat Tapping/$*.fsf ;\

\end{lstlisting}
This block of code runs Feat for the two fMRI blocks. We use the last
cope (cope 4) as the target file to indicate that the recipe has
completed correctly. We set variables for the number of volumes, the
run name, and the run number, and use these (in addition to the
subject and subject directory) to modify the Feat template before
running Feat.



\begin{lstlisting}
	%*\lnote*FirstLevelReg:  FirstLevelFeats $(allupdatereg) 
	$(patsubst %,Tapping/%.feat/reg_standard/mask.nii.gz, $(BLOCKS)) 
	$(patsubst %,Tapping/%.feat/reg_standard/mean_func.nii.gz, $(BLOCKS)) 
	$(patsubst %,Tapping/%.feat/reg_standard/example_func.nii.gz, $(BLOCKS)) 
	$(allupdatetdof) $(patsubst %,Tapping/%.feat/reg/standard.nii.gz, 
	$(BLOCKS)) $(patsubst %,Tapping/%.feat/reg/example_func2standard.mat, $(BLOCKS)) 

	%*\lnote*define make-cope
	Tapping/%.feat/reg_standard/stats/$(notdir $(1)): Tapping/%.feat/stats/$(notdir $(1)) xfm_dir/T1_to_mni_Warp.nii.gz 
	xfm_dir/T1_to_mni_Affine.txt xfm_dir/%_to_T1_ras.txt
		mkdir -p %*\`{}*dirname $$@%*\`{}* ;\
		export ANTSPATH=$(ANTSpath) ;\
		$(ANTSpath)/WarpImageMultiTransform 3 $$(word 1,$$^) $$@ -R $(STD_BRAIN) \
		$$(word 2,$$^) $$(word 3,$$^) $$(word 4,$$^)
	endef

	%*\lnote*$(foreach c,$(allupdatereg),$(eval $(call make-cope,$c)))
\end{lstlisting}
After running Feat, we need to create all the standard space files
that Feat normally would create, using ANTs registration instead of
FEAT to do so. \lnum{14} This target defines all the files that need to be created for Feat using
ANTs registration matrices. We make liberal use of pattern
substitution here to form these names. \lnum{15} Because we have
multiple fMRI runs (TappingBlock1 and TappingBlock2) and a lot of
copes that we wish to register to standard space, we cannot simply use
implicit rules to describe how to transform them. It would be nice if
you could use one character to substitute for the tapping block name,
and another character for the cope number, but \maken{} doesn't work
that way. So you could use a pattern to substitute for the tapping
block name and explicitly write out rules for each of the files to
register. 

Instead, we use a combination of canned recipes, the \texttt{call}
function, and the \texttt{eval} function to create the registration
rules for the copes and varcopes "on the fly" in \maken{}. The first
step here is to define the canned recipe \texttt{make-cope}, which
uses the filename of its first argument to create the snippet of
recipe that registers the cope or varcope file in the stats directory
to standard space. Note that there are two dollar signs for every one
in the recipe. This is necessary because of how this recipe will be
evaluated. \lnum{16} This line does the magic of creating new makefile
rules out of our \texttt{make-cope} variable. The \texttt{foreach}
function goes through the list of files in the \texttt{allupdatereg}
variable that we defined earlier, calling each of them \texttt{c}. For
each file, it then evalutes (using \texttt{eval}) the output of
calling \texttt{make-cope} on each of the files.  The \texttt{call}
function creates a \maken{} recipe for each file, and then
\texttt{eval} causes \maken{} to actually evaluate these new rules, as
if they had been written out in the Makefile from the beginning. 
The argument to \texttt{eval} is expanded twice: first by
\texttt{eval} and then by \texttt{make}, which is why there are extra
\texttt{\$} characters. 

This is long and a little tricky but it does the job.

\begin{lstlisting}
	Tapping/%.feat/reg_standard/example_func.nii.gz: Tapping/%.feat/example_func.nii.gz xfm_dir/T1_to_mni_Warp.nii.gz xfm_dir/T1_to_mni_Affine.txt xfm_dir/%_to_T1_ras.txt
		mkdir -p %*\`{}*dirname $@%*\`{}*;
		$(ANTSpath)/WarpImageMultiTransform 3 Tapping/$*.feat/example_func.nii.gz $@ -R $(STD_BRAIN) xfm_dir/T1_to_mni_Warp.nii.gz xfm_dir/T1_to_mni_Affine.txt xfm_dir/$*_to_T1_ras.txt

	Tapping/%.feat/reg_standard/mean_func.nii.gz: Tapping/%.feat/mean_func.nii.gz xfm_dir/T1_to_mni_Warp.nii.gz xfm_dir/T1_to_mni_Affine.txt xfm_dir/%_to_T1_ras.txt
		mkdir -p %*\`{}*dirname $@%*\`{}*;
		$(ANTSpath)/WarpImageMultiTransform 3 Tapping/$*.feat/mean_func.nii.gz \
		$@ -R $(STD_BRAIN) xfm_dir/T1_to_mni_Warp.nii.gz \
		xfm_dir/T1_to_mni_Affine.txt xfm_dir/$*_to_T1_ras.txt

	Tapping/%.feat/reg_standard/mask.nii.gz: Tapping/%.feat/mask.nii.gz xfm_dir/T1_to_mni_Warp.nii.gz xfm_dir/T1_to_mni_Affine.txt xfm_dir/%_to_T1_ras.txt
		mkdir -p %*\`{}*dirname $@%*\`{}*;
		$(ANTSpath)/WarpImageMultiTransform 3 Tapping/$*.feat/mask.nii.gz \
		$@ -R $(STD_BRAIN) xfm_dir/T1_to_mni_Warp.nii.gz \
		xfm_dir/T1_to_mni_Affine.txt xfm_dir/$*_to_T1_ras.txt

	Tapping/%.feat/reg/standard.nii.gz: xfm_dir/T1_to_mni_Warp.nii.gz
		mkdir -p %*\`{}*dirname $@%*\`{}*;
		cp $< $@

	Tapping/%.feat/reg/example_func2standard.mat: $(TEMPLATES)/selfreg.mat
		mkdir -p %*\`{}*dirname $@%*\`{}*;
		cp $< $@
\end{lstlisting}
There are a variety of files that need to be registered using ANTs
that can easily be specified using implicit rules parameterized by the
block name. We do not need to use \texttt{eval} to create these rules;
we can just write them. 

\begin{lstlisting}
	define make-tdof 
	Tapping/%.feat/reg_standard/stats/$(notdir $(1)): 
	Tapping/%.feat/stats/cope1.nii.gz Tapping/%.feat/stats/dof 
		mkdir -p Tapping/$$*.feat/reg_standard/stats ;\
		fslmaths Tapping/$$*.feat/stats/cope1.nii.gz -mul 0 \
		-add %*\`{}*cat Tapping/$$*.feat/stats/dof%*\`{}* $$@
	endef 

	$(foreach c,$(allupdatetdof),$(eval $(call make-tdof,$c))) 
\end{lstlisting}
Just like the copes, the degrees of freedom files (dof) can be 
created by defining a canned parameterized recipe to create 
them. These files are just NIfTI files of the same dimensions as the 
cope with the degrees of freedom of the analysis in all voxels.  The 
parameterized recipe takes the dof filename as an argument, uses the \texttt{call}
function create a \maken{} recipe for each dof file, and then use 
\texttt{eval} to have \maken{} evaluate these new rules!

\begin{lstlisting}
	SecondLevelFeats: Tapping/Tapping_FixedFX.gfeat/cope4.feat/stats/zstat4.nii.gz

	Tapping/Tapping_FixedFX.gfeat/cope4.feat/stats/zstat4.nii.gz:
        $(Tapping_FEAT2_TEMPLATE) $(patsubst % ,Tapping/%.feat/mask.nii.gz,$(BLOCKS)) FirstLevelReg
		rm -rf `echo $@ | awk -F "/" '{print $$1"/"$$2}'` ;\
		featName=`echo $@ | awk -F "/" '{print $$2}' | \
		awk -F ".gfeat" '{print $$1}'` ;\
		sed -e 's|SUBDIR|$(SUBJECT_DIR)|g' $(word 1,$^) \
		> Tapping/$${featName}.fsf ;\
		feat Tapping/$${featName}.fsf
\end{lstlisting}
After all the ANTs registrations have been applied we can run the
second level Feats. We use the last zstat file as a marker that
indicates Feat has completed successfully. As in the first level Feat, we modify a template
to specify the subject and the block.

\begin{lstlisting}
	clean_feat:
		rm -rf Tapping/*.*feat Tapping/*.fsf
\end{lstlisting}

Finally, we define a target to clean up all files that we have
created! 

The next step in processing, not covered in this example, would be to
perform group level analysis across all subjects.