\section{Testsubject Main Makefile}
\def\sectionautorefname{Testsubject Main Makefile}
\label{sec:testsubjectxfm}

This is an example of a subject-specific makefile that includes
pipelines (written using \maken{}) that have been developed by several people to process different types of
subject-level MRI data. Because there is only one subject in
this example, this subject specific makefile is not a symbolic link,
as it is in the \texttt{oasis-longitudinal-sample-small} directory. 

The code for this example is in \texttt{\$MAKEPIPELINES/testsubject/test001/Makefile}. 

\setcounter{codehighlight}{0} % RESET THIS BEFORE EVERY LST LISTING
\begin{lstlisting}
	%*\lnote*.PHONY = etiv fast flex robex freesurferskullstrip 

	%*\lnote*FSL_DIR=/usr/share/fsl/5.0 

	STD_BRAIN=$(FSL_DIR)/data/standard/MNI152_T1_2mm.nii.gz 

	ifeq "$(origin MAKEPIPELINES)" "undefined"
	MAKEPIPELINES=/project_space/makepipelines 
	endif 

	%*\lnote*t1subj := $(shell pwd) 
	subject := $(notdir $(t1subj)) 

	PROJECT_HOME=$(MAKEPIPELINES)/testsubject/

	%*\lnote*include $(PROJECT_HOME)/lib/makefiles/help_system.mk 
	include $(PROJECT_HOME)/lib/makefiles/resting.mk 
	include $(PROJECT_HOME)/lib/makefiles/xfm.mk 
	include $(PROJECT_HOME)/lib/makefiles/fcconnectivity.mk 
	include $(PROJECT_HOME)/lib/makefiles/QA.mk 

	%*\lnote*export OMP_NUM_THREADS=1 

	SHELL=/bin/bash 

	%*\lnote*FLEXPATH=$(PROJECT_HOME)/bin/wmprogram/sb/cross_platform/scripts

\end{lstlisting}

\lnum{1} We begin by specifying our phony targets. These will be
defined later. 
\lnum{2} This line that sets \texttt{FSL_DIR} will be commented out in
your makefile, but you should find out where your version of FSL is
installed and make sure that it is correct. If we do not specify the
version of the programs that we run in a makefile, \maken{} will use
the individual's \texttt{PATH} variable to find them. Because
different people may have different \texttt{PATH} variables, this will
result in unpredictable results (the opposite of reprodicibility!)
Recall that \maken{} inherits variables from your environment, and you
can override them by setting them.

\lnum{3} It is useful to have a variable to refer to the
subject. Here, we simply obtain the subject from the last part of the
directory name. This won't work for more complicated linking
structures as described in \autoref{sec:practicum2}. 

\lnum{4} We include lots of other makefiles that we need. This keeps
this makefile short and readable. 

\lnum{5} Certain programs do their own parallelization on computers
with multiple processing elements (cores). To turn this off, we
override the environment variable \texttt{OMP_NUM_THREADS}. This
allows us to use the \texttt{-j} flag to \maken{} to parallelize
execution of the entire makefile.

\lnum{6} FLEX is a program for white matter hyperintensity
quantification that we use for this project. This variable simply
specifies its location. If you do not have it installed you just won't
be able to run that part of the processing. It is not mandatory. 

\begin{lstlisting}
	 %*\lnote*FLAIR = $(shell if [ -f 00_NOFLAIR ] ; then echo false; else echo true; fi)

	 HAVET1 = $(shell if [ -f 00_NOT1 ] ; then echo false; else echo true; fi)

	 %*\lnote*ifeq ($(HAVET1),true)
	 %*\lnote*all: $(call print-help,all,Do skull stripping, etiv, HC volume calculation) T1_skstrip.nii.gz first etiv
	 else
	 all:  $(call print-help,all,Do skull stripping, etiv, HC volume calculation)
		@echo "Subject is missing T1 - nothing to do here."
	 endif
\end{lstlisting}

\lnum{7} In large and complicated studies, a certain type of image may
be missing, but this does not stop us from processing all the data
that we have. Here, we test for a FLAIR acquisition by looking for a
marker file in the directory called \texttt{00_NOFLAIR} (and similarly
for the T1). 

\lnum{8} Here we modify the target \texttt{all} depending on whether
or not the subject has a T1 image. We also use the help system
described in \autoref{sec:practicum4} to document what this target
does. 


\begin{lstlisting}
T1_skstrip.nii.gz: T1.nii.gz 
	bet $< $@ -B

robex: $(call print-help,robex,Alternate skull stripping with ROBEX) T1.nii.gz 
	$(PROJECT_HOME)/bin/ROBEX/runROBEX.sh T1.nii.gz T1_skstrip.nii.gz

freesurferskstrip: $(call print-help,freesurferskullstrip, Alternate skull stripping with FreeSurfer) $(PROJECT_HOME)/freesurfer/$(subject)/mri/brain.mgz
	subj=$(subject) ;\
	mri_vol2vol --mov $(PROJECT_HOME)/freesurfer/$${subj}/mri/brain.mgz --targ $(PROJECT_HOME)/freesurfer/$${subj}/mri/rawav
g.mgz --regheader --o brain-in-rawavg.mgz ;\
	mri_convert brain-in-rawavg.mgz brain-in-rawavg.nii.gz ;\
	fslreorient2std brain-in-rawavg.nii.gz T1_skstrip.nii.gz ;\
\end{lstlisting}

We have rules in this makefile for three methods of skull
stripping. The default is simply to call \texttt{bet} with the bias
correction option (which worked well on the data for the study this
makefile was modeled after). However, when this method did not work
well, we tried alternative methods using ROBEX and FreeSurfer. Note
that because the targets \texttt{robex} and \texttt{freesurferskstrip}
are phony, they can be created at any time and will always overwrite
\texttt{T1_skstrip.nii.gz}. 


\begin{lstlisting}
	etiv: $(call print-help, etiv, Estimation of ICV using enigma protocol) eTIV.csv

	brain_to_std.mat brain_to_std.nii.gz: T1_skstrip.nii.gz 
		flirt -in $< -ref $(STD_BRAIN) -omat $@ -out brain_to_std.nii.gz

	eTIV.csv: brain_to_std.mat
		%*\lnote*etiv=`$(PROJECT_HOME)/bin/mat2det brain_to_std.mat | awk '{print $$2 }'` ;\
		echo  $(subject)","$$etiv > $@
\end{lstlisting}

We estimate intracranial volume (ICV) using the ENIGMA protocol (as described in
\autoref{sec:practicum3}). This approach calculates the inverse 
determinant of the linear transformation of the T1 image to standard
space. This is a scaling factor that we can multiply the volume of the
standard space brain by to obtain an estimated ICV volume. 

\lnum{9} note that when calling the \texttt{mat2det} script and
echoing the results to a file, we need two dollar signs (\texttt{\$\$})
for every one that we intend to pass to the shell.  



\begin{lstlisting}
	 first: first_all_fast_firstseg.nii.gz T1.nii.gz hippo.csv

	 first_all_fast_firstseg.nii.gz : T1.nii.gz
		$(PROJECT_HOME)/bin/run_first_all_edit -s "L_Hipp,R_Hipp"  -d -i T1.nii.gz -o first

	 hippo.csv: first_all_fast_firstseg.nii.gz 
		rh=`fslstats $< -u 54 -l 52 -V|awk '{print $$2}'` ;\
		lh=`fslstats $< -u 18 -l 16 -V|awk '{print $$2}'` ;\
		echo $$lh $$rh > hippo.csv
\end{lstlisting}

We run FSL FIRST to calculate hippocampal volumes. We put these into a
comma separated value file to make it easy to remember how to extract
these numbers from the \texttt{first_all_fast_firstseg.nii.gz} and
check them (or include them in a QA report) although there is no
reason you could not write a separate program to gather them all.


\begin{lstlisting}
	 ifeq ($(FLAIR),true)
	 flex: $(call print-help,flex, Run flex for white matter hyperintensity quantification) flair.nii.gz flair_skstrip.hdr flair_skstrip_flwmt_lesions.hdr wmhstats.csv

	 flair_restore.nii.gz: flair.nii.gz
		fast -B -o flair -t 2 $<

	 flair_skstrip.nii.gz: flair_restore.nii.gz
		bet $< $@ -R

	 flair_skstrip.hdr: flair_skstrip.nii.gz 
		fslchfiletype ANALYZE $< $@

	 flair_skstrip_flwmt_lesions.hdr: flair_skstrip.hdr
		@echo "Flex processing " $< 
		$(FLEXPATH)/sb_flex -fl $< 

	 flair_skstrip_renamed.nii.gz: flair_skstrip.nii.gz
		cp $< $@

	 flair_wmh_mask.nii.gz: flair_skstrip_flwmt_lesions.hdr
		fslmaths $< -uthr 1 $@

	 else
	 flex:
		@echo No FLAIR, nothing to do
	 endif

	 wmhstats.csv: flair_skstrip_flwmt_lesions.hdr flair_skstrip_renamed.nii.gz
		@echo Writing wmhstats.csv 
		tot=`fslstats  flair_skstrip_renamed.nii.gz -V | awk '{print $$2}'`; \
		wmh=`fslstats  flair_skstrip_flwmt_lesions.hdr -u 2 -V | awk '{print $$2}'` ; \
		per=`echo $$wmh $$tot | awk '{print ($$1/$$2)*100}'` ; \
		echo  $(subject)", "$$wmh", " $$per >> $@ 
\end{lstlisting}

These are rules to run FLEX, a program for white matter hyperintensity
quantification from FLAIR images. This example comes from a multi-site
study where not all subjects obtained a viable FLAIR scan. If they do
have a FLAIR scan, we bias correct, skull strip, and process the data
using the program \texttt{sb_flex}.  elsewhere). The volume of white
matter hyperintensities (absolute volume and percent of skull stripped
brain) are written to \texttt{wmhstats.csv}. If we don't have a FLAIR
scan, there is nothing to do. You may not have FLEX installed on your
system in which case you can disregard this part of the makefile.


\begin{lstlisting}
	 archive:
		rm -rf flair_bc_*
		rm -rf flair_skstrip_axcor* flair_skstrip_flf_* flair_skstrip_WMT*
		rm -rf *~ \#* brain_to_std.nii.gz flair_skstrip.hdr flair_skstrip.img

	 clean:  $(call print-help,clean,Clean up everything from all makefiles) clean_rest clean_qa clean_transform clean_fcconnectivity
		rm -rf flair_skstrip* flair_pve* flair_restore* *~ wmhstats.csv eTIV.csv brain_to_std* flair_mixeltype.nii.gz T1_skstrip_mask.nii.gz first-L_Hipp* first-R_Hipp* T1_to_std_sub* T1_skstrip.nii.gz hippo.csv first*
\end{lstlisting}

Finally, we define two targets to help us clean up. The first target,
\texttt{archive}, is intended to remove files that we don't need when
we intend to "tidy up". What we remove here is very subjective and
dependent upon the needs of the researchers. 

The second target, \texttt{clean} is intended to remove everything
from all makefiles to clean up everything. Note that \texttt{clean}
depends upon targets such as \texttt{clean_transform} and
\texttt{clean_rest} that are defined in the makefiles we have included
at the beginning. 

