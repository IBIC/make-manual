\Echapter{Downloading Data From XNAT}{Karl Woelfer}{kwoelfer@uw.edu}
\label{chap:XNAT}

This is an example of how to use a Makefile to create and populate a
project directory with images from an open dataset stored in an XNAT
(eXtensible Neuroimaging Archive Toolkit) database, in this case from
the NITRC (Neuroimaging Informatics Tools and Resources Clearinghouse)
1000 Functional Connectomes project image repository at
\url{http://www.nitrc.org/ir}. The code for this example is located at \texttt{fcon_1000/Makefile}.

To run this pipeline you will need to have first created an individual
user account on NITRC, at
\url{http://www.nitrc.org/account/register.php}, and obtained access
to the 1000 Functional Connectome project.

To simplify this example, we download only a subset of the subjects in
the repository. A file with the names of these subjects is used to
determine what files to download.

In our example, we selected 23 Baltimore subjects, saved in the file \texttt{Subjects_Baltimore}.
To recreate this file you can do the following. From the NITRC Image Repository home page ``Search,'' choose Projects $\rightarrow$ 1000 Functional Connectomes. At the 1000 Functional Connectomes project home page, select Options $\rightarrow$ Spreadsheet to download a CSV file of the 1288 subjects \texttt{AnnArbor_sub00306 \ldots Taipei_sub91183}. Select a subset, or all, of the subjects from the first column of the downloaded spreadsheet, and save to a text file. 

The directory hierarchy that this Makefile creates under your project home, from which \maken{} is run, is shown in \autoref{fig:xnat-dir}. Note
that this Makefile assumes that the directory \texttt{visit1} already exists.

\begin{figure}
	\dirtree{%
		.1 fcon_1000/.
		.2 subjects/.
		.3 Baltimore_sub17017/.
		.4 visit1/.
		.5 rest.nii.gz.
		.5 T1_brain.nii.gz.
		.5 T1.nii.gz.
		.3 Baltimore_sub19738/.
		.4 visit1/.
		.5 rest.nii.gz.
		.5 T1_brain.nii.gz.
		.5 T1.nii.gz.
		.3 \ldots.
		.2 visit1/.
		.3 Baltimore_sub17017/\DTcomment{A symlink to ../subjects/Baltimore_sub17017/visit1}.
		.3 Baltimore_sub19738/.
		.3 \ldots.
	}
		\caption{XNAT access directory structure.}
			\label{fig:xnat-dir}
\end{figure}


\begin{lstlisting}
	# Site hosting XNAT
	%*\lnote*NITRC=http://www.nitrc.org/ir
	
	%*\lnote*SHELL=/bin/bash
	
	# Obtain the list of subjects to retrieve from NITRC
	%*\lnote*SUBJECTS = $(shell cat Subjects_Baltimore)

	%*\lnote*.PHONY: clean all allT1 allT1_brain allrest allsymlinks
\end{lstlisting}

This portion of the Makefile defines key variables and targets. 

\indent \lnum{1}We set the ``base name'' of the XNAT web site in a variable \texttt{NITRC}. This can be changed when using another XNAT repository, and the variable can be named accordingly. \\
\indent \lnum{2} By default, \maken{} uses \texttt{/bin/sh} to interpret recipes. Sometimes this can cause confusion, because \texttt{sh} has only a subset of the functionality of \texttt{bash}. We set the \maken{} variable \texttt{SHELL} explicitly. \\
\indent \lnum{3}The \texttt{SUBJECTS} variable will contain a list of the subject data we wish to download. The individual subject names will be used to create directory names. \\
\indent \lnum{4}We define six targets that do not correspond to files, so these are denoted as phony targets.

\begin{lstlisting}
	%*\lnote*all: sessionid allT1 allT1_brain allrest allsymlinks
	%*\lnote*allT1: $(SUBJECTS:%=subjects/%/visit1/T1.nii.gz)
	%*\lnote*allT1_brain: $(SUBJECTS:%=subjects/%/visit1/T1_brain.nii.gz)
	%*\lnote*allrest: $(SUBJECTS:%=subjects/%/visit1/rest.nii.gz)
	%*\lnote*allsymlinks: $(SUBJECTS:%=visit1/%)
\end{lstlisting}

\indent \lnum{5} \texttt{all} is the default target, and simply defines the five dependencies. \\
\indent \lnum{6} We need to derive the names of all the files that we
intend to create from the list of subjects. We do this using pattern
substitution to define several targets. Here, we use pattern matching
to generate a list of the T1 image names that we need to create. Each of
the subject names, e.g. \texttt{Baltimore_sub17017}, is used to create
the corresponding T1 image (e.g. \texttt{subjects/Baltimore_sub17017/visit1/T1.nii.gz}).\\
\indent \lnum{7} As above, we form the names of the skull stripped T1 images.\\
\indent \lnum{8} And the resting state images.\\
\indent \lnum{9} The last thing the Makefile does is create a
\texttt{visit1/} directory after the \texttt{subjects/} directory has
been populated. Here we use pattern matching, as above, to generate
the list of symbolic links that need to be created.  Each
\texttt{visit1/SUBJECT} directory will be a symbolic link to the actual \texttt{subjects/SUBJECTS/visit1/} directory.


\begin{lstlisting}
	%*\lnote*sessionid: 
		@echo -n "Username: " ;\
		read username ;\
		curl --user $$username $(NITRC)/REST/JSESSION > $@
\end{lstlisting}

\indent \lnum{10} Here we use the client URL Request Library
(\texttt{cURL}) to create a session with the XNAT server. The first
line prompts for the user's name on the XNAT server, the second line
reads and stores that in the variable \texttt{username}. With one
single REST transaction, the \texttt{cURL} call on the following line,
we authenticate with the XNAT server, entering a password only once,
and saving the return value \texttt{SESSIONID} in a file named
\texttt{sessionid}. This single session will persist for an
hour. Obtaining a session identifier is important to reduce load on
the remote XNAT server.
	
\begin{lstlisting}	
	%*\lnote*subjects/%/visit1/T1.nii.gz:  | sessionid
	%*\lnote*	mkdir -p `dirname $@`; \
	%*\lnote*	curl --cookie JSESSIONID=`cat sessionid` $(NITRC)/data/projects/fcon_1000/subjects/$*/experiments/$*/scans/ALL/resources/NIfTI/files/scan_mprage_anonymized.nii.gz > $@

\end{lstlisting}

\indent \lnum{11} This recipe downloads all of the T1 images required
by target \texttt{allT1} (\lnum{6}). It has an order-only dependency
upon the file \texttt{sessionid} because we assume that if these files
exist, it does not matter if they are older than the
\texttt{sessionid} file. They will only be recreated if they do not
exist. \\
\indent \lnum{12} This command creates the directory
\texttt{subjects/SUBJECT/visit1} if it does not exist (where SUBJECT
is the actual subject identifier).\\
\indent \lnum{13} This \texttt{curl} command uses the
\texttt{SESSIONID} which was stored in the \texttt{sessionid}
file. The URL defined here is specific to the location where scan data
of interest is stored on the NITRC instance of XNAT. Note that
\texttt{\$*} is used in two places to refer to the subject identifer, denoted by
\texttt{\%} in the target. The XNAT file \texttt{scan_mprage_anonymized.nii.gz} is downloaded and saved under the local name \texttt{T1.nii.gz}.

\begin{lstlisting}
	$subjects/%/visit1/T1_brain.nii.gz:  | sessionid
		mkdir -p `dirname $@`; \
		curl --cookie JSESSIONID=`cat sessionid` (NITRC)/data/projects/fcon_1000/subjects/$*/experiments/$*/scans/ALL/resources/NIfTI/files/scan_mprage_skullstripped.nii.gz > $@
\end{lstlisting}

\indent This recipe is analogous to the previous one, except
that it creates the skull stripped T1 images needed by target
\texttt{allT1_brain}.

\begin{lstlisting}
	subjects/%/visit1/rest.nii.gz:  | sessionid
		mkdir -p `dirname $@`; \
		curl --cookie JSESSIONID=`cat sessionid` $(NITRC)/data/projects/fcon_1000/subjects/$*/experiments/$*/scans/ALL/resources/NIfTI/files/scan_rest.nii.gz > $@
\end{lstlisting}

\indent Similarly, this recipe creates the resting state
images needed by \texttt{allrest}.


\begin{lstlisting}
	visit1/%:
		ln -s ../subjects/$*/visit1 $@
\end{lstlisting}

\indent  This recipe populates the project top-level \texttt{visit1/} directory with symbolic links, pointers to the actual locations of the subjects' \texttt{visit1} data downloaded above. This enables an alternate way to access the subject data.

\begin{lstlisting}
	clean:
		rm -rf subjects; \
		rm -rf visit1/*; \
		rm -f sessionid
\end{lstlisting}

This \texttt{clean} recipe will delete everything in
\texttt{subjects/}, links in the \texttt{visit1/} directory, and the
\texttt{sessionid} file.

