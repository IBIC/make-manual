\chapter*{Practicum 4: TITLE }
\label{sec:practicum4}

\section{The \texttt{Clean} Target}
In \maken{}, the \texttt{Clean} target serves to delete all unnecessary files that were created in the process of running \maken{}. Remember that memory is limited! Cleaning up your project directory will make space for future projects that you will be working on. The target itself is usually specified last in a typical \maken{} recipe, and must be included in your list of \texttt{.PHONY} targets if you choose to run it. 

Now, go to your \texttt{testsubject} directory and open \texttt{Makefile}. Scroll all the way down. You will see both an \texttt{archive} and \texttt{clean} target. \texttt{archive} is a target intended to clean up the directory for archiving after a paper has been accepted. The purpose of this target is to retain important results but remove the partial products. Obviously, what you may define as `important results' depend upon the kinds of questions that might come up later. 

% Natalie: Tara, what are some examples of files that you consider as important? 

The \texttt{clean} target in this makefile includes \texttt{qaclean} and \texttt{restingclean} which are themselves targets in other makefiles in the \texttt{lib} directory. Typically, files that are `easy' to make and are no longer necessary can be removed. Other files, such as those generated by FreeSurfer's recon-all, are not as easily remade. You should therefore think carefully about what you want to remove and what you want to keep.  

\section{Creating a \maken{} Help System}
It is sometimes useful to know what a makefile does without having to look through the entire makefile itself. One way of getting that information at a glance is to create a \maken{} help system that prints out the targets and a short summary explaining what each target does. 

To go about creating this help system, you need to create a separate `help' makefile that tells \maken{} what to do when you call for help. Before we expand on that, however, let us take a look at the main makefile in the \texttt{testsubject} directory copied over from \texttt{project_space}.
\bashcmd{cd ~/testsubject/testsubject}

Open up \texttt{Makefile}. You will notice that one of first few lines at the top of the file asks \maken{} to include a makefile called \texttt{help_system.mk} -- this is the `help' makefile that we alluded to earlier. The \texttt{include} directive tells \maken{} to read other makefiles before continuing with the execution of other things in the current makefile.  

As you scroll down, you will see that some of the targets are followed by a \texttt{call} command before their dependencies. For instance, look a the line:
\makecmd{robex: \$(call print-help,robex,Alternate skull stripping with ROBEX) T1.nii.gz}

This tells \maken{} to return the target \texttt{robex} along with its description when \texttt{print-help} is called (\texttt{print-help} is defined in \texttt{help_system.mk}, as we will see later). Other targets such as \texttt{freesurferskstrip} and \texttt{etiv} also make calls to \texttt{print-help}.

Now, have a look at the \texttt{help_system.mk} file under the \texttt{lib\textbackslash makefiles} directory.  
\makecmd{help: ; @echo \$(if \$(need-help),,Type \'\$(MAKE)\$(dash-f) help\' to get help)}

This line tells \maken{} to echo the fellowing when you type \maken{} into the command line.
\begin{bash}
\$ make
Type `\maken{} help' to get help
\end{bash}

The variable \texttt{need-help} is set such that \maken{} will filter for the word \texttt{help} in your command line entry to decide whether or not you need help. If so, the variable \texttt{print-help} will be called upon -- and this will result in \maken{} printing the name of your targets and their descriptions to your shell. Depending on whether or not this will be helpful for you, you may want to copy \texttt{help_system.mk} to your own project directory and include it in your top-level Makefile. 

We will not delve into the rest of \texttt{help_system.mk} in this practicum. If you are seeking more information on how to create a help makefile, please refer to page 181 of John Graham-Cumming's GNU Make Book. 

\section{Incorporating R Markdown into \maken{}}
Quality assurance is an important step in a neuroimaging analyses pipeline. At IBIC, data is typically  preprocessed and checked for quality before proceeding with further analyses. This can be done both qualitatively and quantitatively. Some of the common aspects of data that are looked at include motion outliers, quality of brain registration/normalization to a subjects-specific or standard template, and whether brain segmentation has been performed correctly. 

Since there are several things that have to be inspected during a quality assurance procedure, it is incredibly helpful to have images and data parameters listed on a single page for each subject in a project. These can be fed into PDF documents or HTML pages. The latter is preferred, however, because it gives us the ability to look at moving GIF images. 

Go to the directory \texttt{testsubject/freesurfer/QA} and open up \texttt{QA_check.html} with your internet browser. 
\bashcmd{iceweasel QA_check.html}











