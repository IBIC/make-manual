\chapter*{Practicum 4: Title?? }
\label{sec:practicum4}

\section{The \texttt{Clean} Target}



\section{Creating a \maken{} Help System}
It is sometimes useful to know what a makefile does without having to look through the entire makefile itself. One way of getting that information at a glance is to create a \maken{} help system that prints out the targets and a short summary explaining what each target does. 

To go about creating this help system, you need to create a separate `help' makefile that tells \maken{} what to do when you call for help. Before we expand on that, however, let us take a look at the main makefile in the \texttt{testsubject} directory copied over from \texttt{project_space}.
\bashcmd{cd ~/testsubject/testsubject}

Open up \texttt{Makefile}. You will notice that one of first few lines at the top of the file asks \maken{} to include a makefile called \texttt{help_system.mk} -- this is the `help' makefile that we alluded to earlier. The \texttt{include} directive tells \maken{} to read other makefiles before continuing with the execution of other things in the current makefile.  

As you scroll down, you will see that some of the targets are followed by a \texttt{call} command before their dependencies. For instance, look a the line:
\makecmd{robex: \$(call print-help,robex,Alternate skull stripping with ROBEX) T1.nii.gz}

This tells \maken{} to return the target \texttt{robex} along with its description when \texttt{print-help} is called (\texttt{print-help} is defined in \texttt{help_system.mk}, as we will see later). Other targets such as \texttt{freesurferskstrip} and \texttt{etiv} also make calls to \texttt{print-help}.

Now, have a look at the \texttt{help_system.mk} file under the \texttt{lib\textbackslash makefiles} directory.  
\makecmd{help: ; @echo \$(if \$(need-help),,Type \'\$(MAKE)\$(dash-f) help\' to get help)}

This line tells \maken{} to echo the fellowing when you type \maken{} into the command line.
\begin{bash}
\$ make
Type `\maken{} help' to get help
\end{bash}




For more information on how to create a help makefile, please refer to the GNU Make cookbook. 

\section{Incorporating R Markdown into \maken{}}



