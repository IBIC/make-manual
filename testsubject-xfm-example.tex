\section{Testsubject Transformations}
%\def\sectionautorefname{Testsubject Transformations}
\label{example:testsubjectxfm}

This is an example of using a makefile to create a set of transformation matrices using different registration methods available in FSL. 

The code for this example is in \texttt{\$MAKEPIPELINES/testsubject/lib/makefiles/xfm.mk}. It is included by \texttt{\$MAKEPIPELINES/testsubject/test001/Makefile}. Therefore, certain variables that this example uses are defined there. This approach helps to organize multiple makefiles and reuse rules across projects.

\setcounter{codehighlight}{0} % RESET THIS BEFORE EVERY LST LISTING
\begin{lstlisting}
	.PHONY=clean_transform transforms 

	transforms:  $(call print-help,xfm,Create resting state to MNI transformations) xfm_dir xfm_dir/MNI_to_rest.mat
\end{lstlisting}

The first line defines two phony targets (clean\_transform and transforms). The \texttt{.PHONY} target can be set as many times as you need to, and note that each makefile included by \texttt{testsubject/test001/Makefile} defines phony targets. 

The second target, \texttt{xfm}, uses the \texttt{print-help} call introduced in \nameref{sec:practicum4} to document this main function, to create an MNI to resting state transformation.

\begin{lstlisting}
	xfm_dir:
	%*\lnote*	mkdir -p xfm_dir

	%*\lnote*xfm_dir/T1_to_MNI.mat: xfm_dir T1_skstrip.nii.gz 
		flirt -in T1_skstrip.nii.gz -ref $(STD_BRAIN) -omat $@
\end{lstlisting}

\lnum{1} We define a target to create a directory, \texttt{xfm_dir},
to hold all of our transformations. This is handy because it allows us to
reuse transformations for other analyses. We know that
the registrations saved here will be checked. 

\lnum{2} This is just a simple rule to call \texttt{flirt} to perform
linear registration of the skull stripped T1 image to the standard
brain. Note that the definition for \texttt{STD_BRAIN} comes from the
including makefile, as do the rules to create the file \texttt{T1_skstrip.nii.gz}.

\begin{lstlisting}
	rest_dir/rest_mc_vol0.nii.gz: rest_dir/rest_mc.nii.gz
		fslroi $< $@ 0 1

	xfm_dir/rest_to_T1.mat: rest_dir/rest_mc_vol0.nii.gz T1_skstrip.nii.gz
		mkdir -p xfm_dir ;\
		%*\lnote*epi_reg --epi=rest_dir/rest_mc_vol0.nii.gz --t1=T1.nii.gz --t1brain=T1_skstrip.nii.gz --out=xfm_dir/rest_to_T1
\end{lstlisting}

These rules use FSL's \texttt{epi_reg} program to register the resting
state data to the subject's structural data. We noticed that
\texttt{epi_reg} used a lot of memory when running, limiting the
number of processors that we could use in parallel to preprocess
resting state data. \lnum{3} This requirement can be circumvented by using only
the first volume of the resting state data, obtained in the first
rule. 


\begin{lstlisting}
	xfm_dir/T1_to_rest.mat: xfm_dir/rest_to_T1.mat
		convert_xfm -omat $@ -inverse $<

	xfm_dir/MNI_to_T1.mat: xfm_dir/T1_to_MNI.mat
		convert_xfm -omat $@ -inverse $<

	xfm_dir/MNI_to_rest.mat:  xfm_dir/T1_to_rest.mat xfm_dir/MNI_to_T1.mat
		convert_xfm -omat xfm_dir/MNI_to_rest.mat -concat xfm_dir/T1_to_rest.mat  xfm_dir/MNI_to_T1.mat
\end{lstlisting}
We obtain the T1 to resting matrix by inverting the resting to T1
matrix, and similarly for the MNI to T1 matrix. Finally, these
matrices are concatenated to create the final target,
\texttt{MNI_to_rest.mat}. Notice that everything else we needed was
automatically created as necessary to make this final target.


\begin{lstlisting}
	clean_transform: 
		rm -rf  xfm_dir 
\end{lstlisting}
Finally, we define a target to remove what we have created and clean
up. Notice that we call it \texttt{clean_transform}, rather than simply
\texttt{clean}, so that it does not override any other targets for
cleaning up that are included by the including Makefile. 