\section{Testsubject FreeSurfer}
\label{example:freesurfer}

This is an example of how to use a makefile to execute FreeSurfer in a cross-sectional context (in contrast to
\nameref{example:freesurfer}), which describes a longitudinal pipeline.  In addition, here we use FreeSurfer to
create a brain mask for skull stripping.

The code for this example is in \texttt{\$MAKEPIPELINES/testsubject/freesurfer/Makefile}.

\setcounter{codehighlight}{0} % RESET THIS BEFORE EVERY LST LISTING

\begin{lstlisting}
	ifeq "$(origin MAKEPIPELINES)" "undefined"
	MAKEPIPELINES=/project_space/makepipelines
	endif

	PROJHOME=$(MAKEPIPELINES)/testsubject 
\end{lstlisting}

This first section of the code looks to see if the environment variable
\texttt{MAKEPIPELINES} is set. This allows people who are not at IBIC
to override the default location of these files.

\begin{lstlisting}
	%*\lnote*include $(PROJHOME)/lib/makefiles/help_system.mk 

	%*\lnote*SUBJECTS=test001 

	export SUBJECTS_DIR=$(PROJHOME)/freesurfer 
	export FREESURFER_SETUP = /usr/local/freesurfer/stable5_3/SetUpFreeSurfer.sh 
	export RECON_ALL = /usr/local/freesurfer/stable5_3/bin/recon-all 
	export TKMEDIT = /usr/local/freesurfer/stable5_3/bin/tkmedit 
	QA_TOOLS=/usr/local/freesurfer/QAtools_v1.1 

	export SHELL=/bin/bash

	%*\lnote*outputstats=$(SUBJECTS:%=%/mri/aparc+aseg.mgz) $(SUBJECTS:%=%/mri/brainmask.mgz)
		$(SUBJECTS:%=%/mri/brainmask.nii.gz)

	%*\lnote*inputdirs = $(SUBJECTS:%=%)

	.PHONY: all clean qa freesurfer

	.SECONDARY: $(inputdirs) $(outputstats)
\end{lstlisting}

This portion of the Makefile defines key variables and targets. 
\lnum{1} We make use of the help system described in
\nameref{sec:practicum4}. 
\lnum{2} There is only one subject here, so for clarity we simply
write it out. However, in real life you would have many subjects, and
obtain them through a wildcard or from a file (see
\nameref{subsec:subjectlist}).
\lnum{3} We use pattern substitution to specify all of the output
targets we wish to create. These are the \texttt{aparc+aseg.mgz} file,
the \texttt{brainmask.mgz} file, and the brainmask converted to NIfTI
format (\texttt{brainmask.nii.gz}). Note that all of these files are
designated as SECONDARY targets so that they are not deleted at the end!
\lnum{4} We also use pattern substitution to set up the input directories.


\begin{lstlisting}
	 all: $(call print-help,all,Setup directories, Run Freesurfer, and Run QA) setup freesurfer qa

	 freesurfer: $(call print-help,freesurfer,Run FreeSurfer) $(outputstats)
\end{lstlisting}

The \texttt{all} and \texttt{freesurfer} targets here are documented
using the help system. 

\begin{lstlisting}
	%/mri/brainmask.mgz: %
		subj=$* ;\
		export FREESURFER_SETUP=$(FREESURFER_SETUP) ;\
		export WATERSHED_PREFLOOD_HEIGHTS='05 15 25 35' ;\
		rm -rf $${subj}/scripts/IsRunning.* ;\
		source $$FREESURFER_SETUP ;\
		export SUBJECTS_DIR=$(SUBJECTS_DIR) ;\
		recon-all  -subjid  $${subj} -multistrip -autorecon1 ;\
		height=`cat $${subj}/mri/optimal_preflood_height' ;\
		if [[ $$height == 05 ]]; then max=$$(( $$height + 5 )); \
		WATERSHED_PREFLOOD_HEIGHTS=`echo $$height $$max'; \
		else min=$$(( $$height - 5 )); max=$$(( $$height + 5 )); \
		WATERSHED_PREFLOOD_HEIGHTS=`echo $$min $$height $$max'; fi ;\
		export WATERSHED_PREFLOOD_HEIGHTS ;\
		recon-all -s $${subj} -multistrip -clean-bm -gcut
\end{lstlisting}

This rule creates the brain mask (in .mgz format). We use the
\texttt{-multistrip} flag to \texttt{recon-all} which allows us to try
multiple preflood heights, trying different thresholds
automatically. Then \texttt{recon-all} is continued using the optimal
height. Note that this approach will begin a separate process
for each preflood height (e.g., four processes). This makes it a bit
hard to run this pipeline on many brains in parallel. Normally, we use
this approach when we are processing subjects one at a time after they
have been scanned.

\begin{lstlisting}
	%/mri/brainmask.nii.gz: %/mri/brainmask.mgz
		export FREESURFER_SETUP=$(FREESURFER_SETUP) ;\
		source $$FREESURFER_SETUP ;\
		mri_convert $< $@
\end{lstlisting}
This rule converts the brain mask from \texttt{mgz} format to NIfTI format.

\begin{lstlisting}
	%/mri/aparc+aseg.mgz:  %/mri/brainmask.mgz 
		subj=$* ;\
		export FREESURFER_SETUP=$(FREESURFER_SETUP) ;\
		source $$FREESURFER_SETUP ;\
		export SUBJECTS_DIR=$(SUBJECTS_DIR) ;\
		recon-all -s $${subj} -autorecon2 -autorecon3 
\end{lstlisting}

This rule finishes the remaining stages of
\texttt{recon-all}. It depends upon the brain mask being generated
(the result of the previous rule).

\begin{lstlisting}
	 setup: $(call print-help,setup,Setup subject directories) $(inputdirs) 

	%:  $(PROJHOME)/%/T1.nii.gz 
		mkdir -p $@/mri/orig; \
		cp $^ $@/mri/orig; \
		cd $@/mri/orig; \
		mri_convert *nii.gz 001.mgz 
\end{lstlisting}

The \texttt{setup} target depends upon the T1 image in the subject
directory (here, in \texttt{\$PROJHOME}). It creates a FreeSurfer
style subject directory, copies the file there, and converts it to
NIfTI format. 

\begin{lstlisting}
	qa: $(call print-help,qa, Run QA - do this interactively with screensaver \
		shut off) $(inputdirs:%=QA/%)

	QA/%: %
		source $(FREESURFER_SETUP) ;\
		$(QA_TOOLS)/recon_checker -s $*
\end{lstlisting}

We run the FreeSurfer QA tools to generate images that we can put into
our own reports. One problem with QA is that the images cannot be
generated in parallel in batch. We include a warning in the help
system that they should be generated interactively without a
screensaver. 

\begin{lstlisting}
clean:
	echo rm -rf $(inputdirs)
\end{lstlisting}

Finally, we define a \texttt{clean} target to remove all processing
results. 
