\Echapter{Seed-based Functional Connectivity Analysis I}{Susan J. Melhorn}{smelhorn@uw.edu}
\label{example:fcconnectivity}
This is an example of how to use a makefile to execute a
single-subject seed-based functional connectivity analysis. Here, the network of interest is the salience network with a 4mm sphere in the right insula as the seed region (\texttt{seedrins4.nii.gz}). Note the seed region mask has already been created based on the literature (\cite{seely2008,lee2014}).

We conduct a seed-based analysis by (1) extracting the time course of
interest, and (2) using this time course as a regressor in a GLM,
including all nuisance variables. Here, this analysis is implemented
using FSL's FEAT (\texttt{feat}) program.

%Lee SE, Khazenzon AM, et al. Altered network connectivity in frontotemporal dementia with C9orf72 hexanucleotide repeat expansion. Brain 2014;137:3047-3060.
%Seeley WW, Crawford R, Rascovsky K, Kramer JH, Weiner M, Miller BL, et al. Frontal paralimbic network atrophy in very mild behavioral variant frontotemporal dementia. Arch Neurol 2008;65:249-55. 

The code for this example is in \texttt{testsubject/lib/makefiles/fcconnectivity.mk}.
Note that the variables \texttt{PROJECT_HOME, FSL_DIR, STD_BRAIN, and subject} are all set elsewhere, but are needed.  This example builds upon the other processing pipelines in place for \texttt{testsubject}.

\begin{lstlisting}
	.PHONY: Connectivity clean_fcconnectivity

	Connectivity: $(call print-help,Connectivity,"Perform subject-level seed-based connectivity analysis") \
	fcconnectivity_dir/Rins_in_func.nii.gz fcconnectivity_dir/Rins_ts.txt \
	fcconnectivity_dir/$(subject).feat/stats/zstat1.nii.gz \
	fcconnectivity_dir/$(subject).feat/stats/FZT1.nii.gz

	%*\lnote*fcconnectivity_dir/Rins_in_func.nii.gz: $(PROJECT_HOME)/lib/fcconnectivity/seedrins4.nii.gz xfm_dir/MNI_to_rest.mat rest_dir/rest_ssmooth.nii.gz
		mkdir -p fcconnectivity_dir ;\
		flirt -in $(PROJECT_HOME)/lib/fcconnectivity/seedrins4.nii.gz \
		-ref rest_dir/rest_ssmooth.nii.gz -applyxfm \
		-init xfm_dir/MNI_to_rest.mat -out $@ ;\
		fslmaths $@  -bin $@ 
	
	%*\lnote*fcconnectivity_dir/Rins_ts.txt: fcconnectivity_dir/Rins_in_func.nii.gz 
				rest_dir/rest_ssmooth.nii.gz
		mkdir -p %*\`{}*dirname $@ %*\`{}* ;\
		fslmeants -i rest_dir/rest_ssmooth.nii.gz -o $@ \
		-m fcconnectivity_dir/Rins_in_func.nii.gz

\end{lstlisting}

\lnum{1}This target puts the seed mask in functional space from
standard space. Note the dependencies here are created in the resting
state preprocessing makefile. Normally we would threshold the mask at
.5 to preserve its volume, but this particular seed is so small that
we only binarize it using \texttt{fslmaths}
\lnum{2} We use \texttt{fslmeants} to extract the timeseries from the
masked functional connectivity data.

\begin{lstlisting}
	%*\lnote*fcconnectivity_dir/fcconnectivity.fsf: $(PROJECT_HOME)/lib/fcconnectivity/fcconnectivityTEMPLATE.fsf
		pipelines=%*\`{}*echo $(MAKEPIPELINES)|sed `s/\//\\\\\//g'%*\`{}*;\
		sed -e `s/SUBJECT/$(subject)/g' \
		-e `s/MAKEPIPELINES/'$$pipelines'/g'  $< > $@
\end{lstlisting}
This step relies on a template \texttt{.fsf} file created using the
\texttt{Feat} graphical user interface. We set this up for the first
subject, and use the text file for the seed time course as an
explanatory variable. The input data is the resting state data that
has been processed up until nuisance regressors are removed. 
We use no convolution (the seed time course is
already convolved by the brain) and turn off temporal derivative and
temporal filtering. We include all nuisance regressors in the model. 

The \texttt{.fsf} file created
in this way has hardcoded paths that include the directory of the
specific subject and the path to all of the pipeline examples. To make
it work for other subjects (at other sites) we parameterize the file
created by \texttt{Feat} to substitute the path to the subject with
the text \texttt{MAKEPIPELINES}, and the subject identifier with the
text \texttt{SUBJECT}. You can see this file at
\texttt{lib/fcconnectivity/fcconnectivityTEMPLATE.fsf}. We use the
\texttt{sed} command to replace these placeholder strings with their
actual values, thus generating a customized \texttt{.fsf} file for
each subject. We set the pipelines variable just to escape the slashes
(/) in the MAKEPIPELINES variable for \texttt{sed}.
If you like, after you run this step, you can open up the newly
created \texttt{.fsf} file to see the settings that are described above.
\begin{lstlisting}		
	fcconnectivity_dir/$(subject).feat/stats/zstat1.nii.gz: 
	rest_dir/rest_ssmooth.nii.gz rest_dir/rest_nuisance_regressors.txt \
	fcconnectivity_dir/Rins_ts.txt fcconnectivity_dir/fcconnectivity.fsf 
		feat fcconnectivity_dir/fcconnectivity.fsf 
\end{lstlisting}

Run \texttt{feat}. Note that there are several dependencies in this
rule that don't appear anywhere else in the recipe. However, they are
needed by the \texttt{.fsf} file used to run \texttt{feat}.

\begin{lstlisting}
	fcconnectivity_dir/$(subject).feat/stats/FZT1.nii.gz: 
	fcconnectivity_dir/$(subject).feat/stats/zstat1.nii.gz \
	rest_dir/rest_ssmooth.nii.gz rest_dir/rest_nuisance_regressors.txt 
		NBVOLS=`fslval rest_dir/rest_ssmooth.nii.gz dim4';\
		NBNUISANCE=%*\`{}*awk `{print NF}' rest_dir/rest_nuisance_regressors.txt \
		|head -1%*\`{}* ;\
		SD=%*\`{}*echo "sqrt($$NBVOLS-$$NBNUISANCE-3)" | bc -l%*\`{}* ;\
		echo $$SD ;\
		fslmaths fcconnectivity_dir/$(subject).feat/stats/zstat1.nii.gz \
		-div $$SD fcconnectivity_dir/$(subject).feat/stats/FZT1.nii.gz 
		
\end{lstlisting}

After running \texttt{feat}, we need to convert the zstat to Fisher's
Z transformed partial correlation. This math is implemented using
\texttt{fslmaths} and is based on Jeanette Mumford's \href{http://mumfordbrainstats.tumblr.com/post/125523326931/how-to-convert-zstat-images-to-fishers-z}{recommendation}.  

\begin{lstlisting}
	clean_fcconnectivity:
		rm -rf fcconnectivity_dir
\end{lstlisting}

Finally, we define a target to remove all the files we have created.

