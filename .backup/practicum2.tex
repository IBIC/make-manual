\chapter{ \maken{} in Context}

\section{Lecture: Organizing Subject Directories }

In practice, a project will involve multiple subjects, each with
several types of scans. One could imagine a universal standard where
every NifTI file contained information about where it came from and
what information it contained, and all neuroimaging programs
understood how to read and interpret this information. We could dump
all of our files into a database and extract exactly what we
wanted. We wouldn't need to use something as archaic as \maken{} because
we could specify dependencies based on the products of certain
processing steps, no matter what they were called.

If you can't imagine this, that's totally fine, because it's a very
long way off and won't look like that when it's here. Right now, we
need to work with UNIX file naming conventions to do our processing.

Therefore, selecting good naming conventions for files and for
directories is key. \maken{} specifically depends upon naming
conventions so that people can keep track of what programs and steps
were used to generate what projects.

% It's ok here to reprint modified info from Chapter 2 

Many of our studies are longitudinal. Even if they don't start out
that way, if you are trying to study a developmental or degenerative
disease, and you scan subjects once, it is often beneficial to scan
them again to answer new questions about longitudinal
progression. However, this aspect of study design poses some
challenges for naming conventions. 

Because of the way that tools like XNAT like to organize data, we
organize multiple visits for each subject as subdirectories under that
subject's main data directory. However, this organization is highly
inconvenient for processing with \maken{}.  


\section{Practical Example: A More Realistic Longitudinal Directory Structure}

\subsection{Organizing Longitudinal Data}

Follow along with this example. Copy directory 
\newline\texttt{/project_space/makepipelines/oasis-longitudinal-sample-small/} \newline
to your home directory. \mymarginnote{!}{Don't forget use \texttt{cp -r} to copy all files in a directory.}

This is a very small subset of the OASIS data set, which consists of a
longitudinal sample of structural images. There are several T1 images
taken at each scan session, and several visits that occurred a year or
more apart. I’ve reduced the size by taking only one of the T1 images
for each person, for each visit, and only one a small sample of
subjects.

Look at the directory structure of \textt{subjects}. A useful command to do this is called \texttt{find}. For example, if you are in the \textt{subjects} directory you can type:
\bashcmd{find .}

You can see that as we have discussed, each subject has one to five
separate sessions. The data for each session (here, only a single
MPRAGE) is stored under each session directory. I realize that
creating a directory to store what is right now a single scan seems a
bit like overkill, but in a real study there would be several types of
scans in each session directory. Here, to focus on the directory
organization and how to call make recursively, we are only looking at
one type of scan.

Normally there are two types of processing that are performed for a
study. The first are subject-level analyses --- in short, analyses that
are completely independent of all the other subjects. The second are
group-level analyses, or analyses that involve all the subjects
data. In general, a good rule of thumb is that the results of
subject-level analyses are best placed within the subject directories.

Group-level analyses seem to be best found elsewhere in the project
directory --- either at a specific timepoint or organized at the top
level.

Create the directory \textt{visit1}. \texttt{cd} into it. Now what we want to do is create the symbolic links for each subject's first visit here.

You can do one by hand:
\bashcmd{ln -s ../subjects/OAS2_0001/visit1 OAS2_0001}

You can also do this in bulk. Remove this link that you created and
use the program in that directory (\textt{makelinks})\mymarginnote{!}{The current directory won't be in your PATH, so make sure to call it with \textt{./makelinks}.} to create all the subject links for visit1. 

Now note that if you do an \textt{ls} it doesn't normally follow the
link. You can do an \textt{ls -l} to look at it, \textt{ls -F} to see
in shorthand that it is indeed a link, or \textt{ls -L} if you want to follow
it. With symbolic links is is very helpful to be able to know where
you are.\mymarginnote{!}{Your \textt{ls} command may be aliased to something pleasing, in which case you might see slightly different behavior than described here.}

\subsection{Recursive \maken{}}
Now let’s look at the Makefile that is there
(\textt{visit1/Makefile}).  The idea here is that this is just a
top-level “driver” makefile for each of the subjects. All it does is
create a symbolic link, if necessary, to the subject-level makefile,
and then it goes in and calls \maken{} in every subdirectory.

Do you know why is the subject target a phony target? The reason is
that we want make to be triggered within the subject directory every
time we call it.

Create the symbolic links to the subject-level makefile as follows:
\bashcmd{make makefiles}

Do you know why we use the \textt{\$PWD} variable to create the link?
If we used a relative path to the target file, what would happen when
we go to the subject directory?

Let us see how this works. Look at the subject-level makefile. Go into
a subject directory and run \textt{make}.

By now you might be getting really tired of seeing those same
\textt{fslreorient2std} and \textt{bet} commands. By default, \maken{}
will echo the commands that it executes to the terminal. If you would
like it to stop doing that, you can tell it not to do that by
prepending the \textt{@} character to each line.
\begin{make}{The \textt{@} character hides commands}
\maker{\%\_T1\_skstrip.nii.gz}{\%\_T1.nii.gz}\\
  \tab @bet $< $@ \\
\end{make}

Now go find the same subject via the subject subtree. Type
\textt{make} and see that it works. Part of this magic is that we set
the subject variable correctly, even though where it appears in a
directory path is different in each place.

There is a useful little rule defined in the GNUMake Cookbook (Chapter
2, p. 44) that may be useful for checking that you have set variables correctly. Add the following lines to the subject-level makefile.


\begin{make}{Printing out the value of variables}
\maker{print-\%}{}
\tab @echo \$* = \$(\$*\)
\end{make}

Now 
