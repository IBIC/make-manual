\chapter{Documenting a makefile for this manual}

This chapter will show you how to document (in \LaTeX) a makefile for this manual. While \LaTeX{} may look intimidating if you haven't used it before, it is fairly straightforward to use. We have created a number of tools to help you insert your makefile into this manual.

\section{Makefile to \LaTeX}

There are a number of things that need to be changed in a bare makefile in order to import it to the \LaTeX{} structure used for this manual. To do the basic stuff, call the script \texttt{makeToTex}\footnote{Please report bugs to Trevor at \url{tkmday@uw.edu}.} on your makefile(s). It can accept any number makefiles and outputs \texttt{<file>_texed.txt}. It has two flags, \texttt{h} and \texttt{v}, which call the help menu and verbose mode, respectively.

There are two main things that need to happen: certain symbols need to be escaped in \LaTeX, and recipes need to be formatted in a  macro to make them pretty. Symbols like ``\$'' and ``\mypound'' have special functions in \LaTeX, and will result in catastrophe if you don't fix them.

The functionality of the script is documented in \autoref{appendix:makeToTex}. Please refer to it if you have any troubles.

Now to bring your new makefile into a \LaTeX document!

\section{Editing a .tex File}

Any script compiled with the main document will include the necessary environment \texttt{makefileread}. In your \LaTeX document, create a newline, ensuring the indentation is consistent and insert the command \verb!\begin{makefileread}!. If you are using a \LaTeX editor, it may autocomplete with the last line, \verb!\end{makefileread}.! If not, you'll have to type it yourself.

Paste your code between those two commands. It should be one level more indented.

To insert text comments between sections of code, you may terminate the environment, add your text and begin a new environment with the rest of the code. Refer to the style guide (\autoref{appendix:sg}) for stylistic choices we use to keep our text consistent. There are many idiosyncrasies to \LaTeX, too many to go into here, but a few of note:

\begin{easylist}[itemize]
	& Quotes are written with two backticks (opening) or two apostrophes (closing).
	& Please leave a blank line between paragraphs. 
	& Special characters should always be escaped (see \autoref{appendix:makeToTex} or Internet resources).
\end{easylist}

Feel free to scroll through chapter files to look for examples of usage if you are in doubt. The answers to further questions can probably be found online (I especially like \url{http://tex.stackexchange.com}).

\section{How Makefile Documentation Should Look}

Notice how documentation will begin on a new page and a new page will be created at the end of the documentations. This is necessary to allow for wider margins that make the code easier to read.

\begin{makefileread}
	
	
\newgeometry{scale=0.85, centering}

PROJHOME=/projects2/act-plus/subjects/session1 \\

cwd=\$(shell pwd) \\

SUBJECTS=\$(shell cat /projects2/act-plus/uds/good_subjects.txt) \\

export OMP_NUM_THREADS=1 \\

export SUBJECTS_DIR=/projects2/act-plus/freesurfer \\
export QA_TOOLS=/usr/local/freesurfer/QAtools_v1.1 \\

\begin{center}
	\begin{minipage}{0.65\textwidth}
		This is some commentary on your makefile
		\lmhrule
	\end{minipage}
\end{center}

export FREESURFER_SETUP = /usr/local/freesurfer/stable5_3/SetUpFreeSurfer.sh \\
export RECON_ALL = /usr/local/freesurfer/stable5_3/bin/recon-all \$(RECON_FLAGS) \\
export TKMEDIT = /usr/local/freesurfer/stable5_3/bin/tkmedit \\

define usage \\
@echo Usage: \\
@echo ``make\, or make interactive\hfill Makes interactive targets'' \\ 
@echo ``make noninteractive\hfill Makes noninteractive targets'' \\ 
@echo \\
@echo Noninteractive targets: \\
@echo ``make setup\hfill Copies source files to this directory'' \\ 
@echo ``make freesurfer\hfill Runs freesurfer'' \\ 
@echo \\
@echo Other useful targets: \\
@echo ``make clean\hfill Remove everything! Be careful!'' \\ 
@echo ``make mostlyclean\hfill Remove everything but the good bits.'' \\ 
@echo ``make help\hfill Print this message.'' \\ 
@echo \\
@echo Variables: \\
@echo ``RECON_FLAGS\hfill Set to flags to recon-all\, by default'' \\ 
@echo ``WAVE\hfill 1\, 2 or 3 to select subjects\, for setup'' \\ 
endef \\

export SHELL=/bin/bash \\

\maker{.PHONY}{qa clean mostlyclean output noninteractive} \\

\maker{noninteractive}{setup freesurfer} \\

\maker{all}{noninteractive} \\

output=\$(SUBJECTS:\%=\%/mri/aparc+aseg.mgz) \$(SUBJECTS:\%=\%/mri/brainmask.nii.gz) \\
\maker{freesurfer}{\$(output)} \\

\maker{qafiles=\$(SUBJECTS}{\%=QA/\%)} \\

\maker{qa}{\$(qafiles)} \\

\maker{QA/\%}{\%} \\
\tab source \$\$FREESURFER_SETUP  ;\textbackslash \\
\tab \$(QA_TOOLS)/recon_checker -s \$* \\

\maker{\%/mri/aparc+aseg.mgz}{\$(PROJHOME)/\%/memprage/T1.nii.gz} \\
\tab rm -rf `dirname \$@`/IsRunning.* \\
\tab source /usr/local/freesurfer/stable5_3/SetUpFreeSurfer.sh  ;\textbackslash \\
\tab export SUBJECTS_DIR=\$(SUBJECTS_DIR)  ;\textbackslash \\
\tab /usr/local/freesurfer/stable5_3/bin/recon-all -i \$< -subjid \$* -all  ;\textbackslash \\
\tab /usr/local/freesurfer/stable5_3/bin/recon-all -s \$* -T2 \$(PROJHOME)/\$*/flair/Flair.nii.gz -T2pial \\

\maker{\%/mri/brainmask.nii.gz}{\$(SUBJECTS_DIR)/\%/mri/aparc+aseg.mgz} \\
\tab source /usr/local/freesurfer/stable5_3/SetUpFreeSurfer.sh  ;\textbackslash \\
\tab mri_convert \$(SUBJECTS_DIR)/\$*/mri/brainmask.mgz \$@ \\

\maker{clean}{} \\
\tab echo rm -rf \$(inputdirs) \\

\maker{mostlyclean}{} \\
\tab @echo Here I would delete things that are not necessary after all is said and done. \\

\maker{output}{} \\
\tab @echo Nothing yet \\

\maker{setup}{\$(SUBJECTS)} \\

\maker{\%}{\$(PROJHOME)/\%/memprage/T1.nii.gz} \\
\tab mkdir -p \$@/mri/orig ;\textbackslash \\
\tab cp \$\textasciicircum \$@/mri/orig ;\textbackslash \\
\tab cd \$@/mri/orig ;\textbackslash \\
\tab mri_convert T1.nii.gz 001.mgz \\

\maker{help}{} \\
\tab \$(usage) \\
\restoregeometry

\end{makefileread}

