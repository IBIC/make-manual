\documentclass[oneside,11pt]{memoir}

% Fonts
\usepackage[T1]{fontenc}	% better output encoding
\usepackage{lmodern}		% Latin Modern instad of CM
\usepackage{inconsolata}	% better monospaced font
\usepackage{microtype}		% fix the little things % comes after any fonts

% basics
\usepackage{graphicx}	% now you can add pretty pictures
\usepackage{url} 		% makes urls look nicer
\usepackage[usenames,dvipsnames]{color} % adds colored text
\usepackage[font=footnotesize]{caption}	% more control over figure captions
\usepackage{xparse}		% nesting arguments wtihin environments
\usepackage{underscore}	% better text underscores, defua
\usepackage{marginnote}	% better margin notes
\usepackage{tabularx}	% more flexible tables
\usepackage[table]{xcolor}	% alternating table colors
\usepackage{todonotes}	% backend organization. Also for some reason necessary for tikz

% Complex packages
\usepackage{tcolorbox} 	% frames
\usepackage{listings} 	% allows for adding $ at the beginning of lines
\usepackage{dirtree}	% pretty directory trees

% Lists
\usepackage[ampersand]{easylist}	% Nicer syntax for creating lists
\usepackage{paralist}	% Allows inline lists

% Hyperref
\usepackage[colorlinks=true]{hyperref} 	% allows hyperlinking % NEEDS TO BE LOADED LAST

% underscore defaults to text mode over math mode
\catcode`_=12 

% Document information
\title{Using GNU \texttt{make} for Neuroimaging Workflow}
\author{ Tara M. Madhyastha (\url{madhyt@uw.edu}) \and Zo\'e Mestre (\url{zlm@uw.edu}) %
	\and Trevor McAllister-Day (\url{tkmday@uw.edu}) \and Natalie Koh (\url{natkoh@uw.edu)} %
}

% Set up the margins and the mymarginnote command
\setlrmargins{*}{*}{1}
\setmarginnotes{0.25in}{1in}{0.5in}
\checkandfixthelayout

\newcommand{\mymarginnote}[2]{\marginpar{{\small \sffamily \selectfont\fbox{\textbf{#1}} #2}}} % custom margin notes

% Code display commands
\newcommand\maken{\texttt{make}} % typeset 'make' always in monospace
\newcommand\bashn{\texttt{bash}} % same for 'bash'

\definecolor{textgray}{gray}{0.75}
\makeatletter
\newcommand{\onelbox}[2]{%
	\setbox0=\hbox{#1}%
	\setlength{\@tempdima}{\dimexpr\wd0+13pt}%
	\begin{tcolorbox}[colback=#2,boxrule=0.5pt,arc=4pt,
		left=6pt,right=6pt,top=6pt,bottom=6pt,boxsep=0pt,width=\textwidth]
		#1
	\end{tcolorbox}
}
\makeatother

% ONE-LINE environments
% $ is automatically added to the bash prompt.
% % \hfill\, forces the minipage onto a new line. Don't ask.
\newcommand{\bashcmd}[1]{ \hfill\, \begin{minipage}[t]{\linewidth}  \hrule \vspace{0.5\baselineskip} \texttt{\small \$ #1} \vspace{0.5\baselineskip} \hrule \end{minipage} \vspace{0.5\baselineskip} }
\newcommand{\makecmd}[1]{ \vspace{0.5\baselineskip} \onelbox{\texttt{\small #1}}{white} \vspace{0.5\baselineskip} }
% Since these commands are embedded in floats, the nicest way to space them properly in the output
% % is to make sure that there is a newline between the command and the text you want to follow the figure.

% load languages for the listings class
\lstloadlanguages{bash,make}

% DEFINE BASH MULTILINE ENVIRONMENT
\makeatletter
% define custom macro that expands to the language name
% (for comparison purposes)
\newcommand\langname@bash{}
\def\langname@bash{bash}
% define custom prompt
\newcommand\prompt@bash{\$\ }
% define a macro (initially empty) and insert it at the beginning of every paragraph
\newcommand\addedToEveryPar@bash{}
%\lst@AddToHook{EveryPar}{\addedToEveryPar@bash}
% redefine the macro by the custom prompt, but only if the language in use
% be `bash'
\lst@AddToHook{PreInit}{%
	\ifx\lst@language\langname@bash%
	\let\addedToEveryPar@bash\prompt@bash%
	\fi
}
\makeatother
\lstnewenvironment{bashsession}[1][]{\lstset{language=bash,gobble=8,basicstyle=\small\ttfamily#1}}{}

\newenvironment{bash}[2] % 
{ \begin{samepage} \begin{figure}[h!] \def\captiontext{#1} \def\labelt{#2} \vspace{0.5\baselineskip} \hrule \vspace{0.5\baselineskip} \ttfamily } %
		{ \vspace{0.5\baselineskip} \hrule \vspace{0.5\baselineskip}  \caption{\captiontext} \label{\labelt} \end{figure} \end{samepage} 	} 
	
% % This is rather unfortunate, but you can't nest \lstnewenvironment inside a \newenvironment so inorder to typset a block of bash code use :
% % 	\begin{bash}\begin{bashsession}
% %			multi-line
% %			code
% %		\end{bashsession}\end{bash}
	
% DEFINE MAKE MULTILINE ENVIRONMENT
\newenvironment{make}[2] %
{ \begin{figure}[h!] \def\captiontext{#1} \def\labelt{#2} \begin{tcolorbox}[colback=white,boxrule=0.5pt,arc=4pt,left=6pt,right=6pt,boxsep=0pt,width=\textwidth] \ttfamily }%
		{ \normalfont \end{tcolorbox} \caption{\captiontext} \label{\labelt} \end{figure} }
% You must terminate lines in the make environment
% To indent lines, use \tab
			
% use \maker to create recipes
\newcommand{\maker}[2]{\textcolor{blue}{#1}:\enspace#2}
			
% tab for indenting in make environment
\newcommand{\tab}{\hspace*{4em}}
			
% Specify command for "--"
% use this to specify bash long inputs (--) so they don't get swallowed up into hyphens.
% No space follows by default
\newcommand{\dd}{-{}-} 
			
% Change numbering
\setsecnumdepth{chapter}	% only number chapters
\maxtocdepth{subsection}	% put subsections in the table of contents
\makeatletter
\@addtoreset{chapter}{part}	% reset the chapter counter at each part
\makeatother
			
\pagestyle{ruled}	% Graphics command
\raggedbottom		% Allow more flexibility at the bottom of the page - helpful with lots of floats.
			
\newcommand{\martt}[1]{\texttt{\color{gray!75} #1}}

\begin{document}

\small\ttfamily

\# act-plus freesurfer makefile \\
	
\# This is where the subject directories live \\
PROJHOME=/projects2/act-plus/subjects/session1 \\
	
cwd=\$(shell pwd) \\
	
\# This sets the subject directories to be the ones we selected \\
SUBJECTS=\$(shell cat /projects2/act-plus/uds/good_subjects.txt) \\
\#SUBJECT=\$(notdir \$(cwd)) \\
	
\# Set open MP number of threads to be 1\, so that we can parallize using make. \\
export OMP_NUM_THREADS=1 \\

\# for Freesurfer (running version 5.3) \\
export SUBJECTS_DIR=/projects2/act-plus/freesurfer \\
export QA_TOOLS=/usr/local/freesurfer/QAtools_v1.1 \\
	
\# be really careful with paths and variables - two versions of freesurfer \\
\# installed \\
export FREESURFER_SETUP = /usr/local/freesurfer/stable5_3/SetUpFreeSurfer.sh \\
export RECON_ALL = /usr/local/freesurfer/stable5_3/bin/recon-all \$(RECON_FLAGS) \\
export TKMEDIT = /usr/local/freesurfer/stable5_3/bin/tkmedit \\
	
define usage \\
@echo Usage: \\
@echo ``make\, or make interactive\tab Makes interactive targets'' \\ 
@echo ``make noninteractive\tab \tab Makes noninteractive targets'' \\ 
@echo \\
@echo Noninteractive targets: \\
@echo ``make setup\tab \tab \tab Copies source files to this directory'' \\ 
@echo ``make freesurfer\tab \tab Runs freesurfer'' \\ 
@echo \\
@echo Other useful targets: \\
@echo ``make clean\tab \tab \tab Remove everything! Be careful!'' \\ 
@echo ``make mostlyclean\tab \tab Remove everything but the good bits.'' \\ 
@echo ``make help\tab \tab \tab Print this message.'' \\ 
@echo \\
@echo Variables: \\
@echo ``RECON_FLAGS\tab \tab \tab Set to flags to recon-all\, by default'' \\ 
@echo ``WAVE\tab \tab \tab \tab 1\, 2 or 3 to select subjects\, for setup'' \\ 
endef \\
	
export SHELL=/bin/bash \\

\maker{.PHONY}{qa clean mostlyclean output noninteractive} \\

\maker{noninteractive}{setup freesurfer} \\

\maker{all}{noninteractive} \\
	
output=\$(SUBJECTS:\%=\%/mri/aparc+aseg.mgz) \$(SUBJECTS:\%=\%/mri/brainmask.nii.gz) \\
\maker{freesurfer}{\$(output)} \\
	
\#recon-all \$(RECON_FLAGS) -subjid  \$\${subj} -FLAIR \$\${subj}/flair.nii.gz -FLAIRpial ;\textbackslash \\
	
\#\#\#\#\#\#\#\#\#\#\#\#\#\#\#\#\#\#\#\#\#\#\#\#\#\#\#\#\#\#\#\#\#\#\#\#\# \\
	
\maker{qafiles=\$(SUBJECTS}{\%=QA/\%)} \\
	
\maker{qa}{\$(qafiles)} \\
	
\maker{QA/\%}{\%} \\
\tab source \$\$FREESURFER_SETUP  ;\textbackslash \\
\tab \$(QA_TOOLS)/recon_checker -s \$* \\
	
\#\#\#\#\#\#\#\#\#\#\#\#\#\#\#\#\#\#\#\#\#\#\#\#\#\#\#\#\#\#\#\#\#\#\#\#\# \\
	
\# No more Freesurfer mixed in with other stuff \\
	
\maker{\%/mri/aparc+aseg.mgz}{\$(PROJHOME)/\%/memprage/T1.nii.gz} \\
\tab rm -rf `dirname \$@`/IsRunning.* \\
\tab source /usr/local/freesurfer/stable5_3/SetUpFreeSurfer.sh  ;\textbackslash \\
\tab export SUBJECTS_DIR=\$(SUBJECTS_DIR)  ;\textbackslash \\
\tab /usr/local/freesurfer/stable5_3/bin/recon-all -i \$< -subjid \$* -all  ;\textbackslash \\
\tab /usr/local/freesurfer/stable5_3/bin/recon-all -s \$* -T2 \$(PROJHOME)/\$*/flair/Flair.nii.gz -T2pial \\
	
\#Use Registration matrix to create skull stripped brain (memprage/T1_brain) from FreeSurfer's skull strip in fsl T1 space \\
	
\maker{\%/mri/brainmask.nii.gz}{\$(SUBJECTS_DIR)/\%/mri/aparc+aseg.mgz} \\
\tab source /usr/local/freesurfer/stable5_3/SetUpFreeSurfer.sh  ;\textbackslash \\
\tab mri_convert \$(SUBJECTS_DIR)/\$*/mri/brainmask.mgz \$@ \\

\maker{clean}{} \\
\tab echo rm -rf \$(inputdirs) \\
	
\maker{mostlyclean}{} \\
\tab @echo Here I would delete things that are not necessary after all is said and done. \\
	
\maker{output}{} \\
\tab @echo Nothing yet \\
	
\maker{setup}{\$(SUBJECTS)} \\

\maker{\%}{\$(PROJHOME)/\%/memprage/T1.nii.gz} \\
\tab mkdir -p \$@/mri/orig ;\textbackslash \\
\tab cp \$\textasciicircum \$@/mri/orig ;\textbackslash \\
\tab cd \$@/mri/orig ;\textbackslash \\
\tab mri_convert T1.nii.gz 001.mgz \\
	
\maker{help}{} \\
\tab \$(usage) \\
	
\end{document}