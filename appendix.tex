
	
\chapter{Style Guide}
\label{appendix:sg}
	
Please refer to this guide when in doubt. It covers typographic and phrasing consistencies.
	
\begin{easylist}[enumerate]
	& Monospace (use the \verb!\texttt{}! macro) the following things:
	&& \maken, when referring to the program.
	&&& Use the macro \verb!\maken{}!. To remove the trailing space (for example, to use a comma after \maken), call the macro without the braces, \verb!\maken!.
	&&& Never uppercase.
	&& \bashn
	&&& Use the macro \verb!\bashn{}!.
	&& Any \bashn{} or \maken{} command. This is done automatically in the one-liner and multi-line environments. Be sure to monospace commands written in-line.
	&& Explicit references to a file: e.g., ``\ldots open \texttt{subjects.txt}.''
	&& Commands run from the command line: e.g., \texttt{ls}, \texttt{bet}, \texttt{flirt}, \texttt{qmake}, etc \ldots
	&& References to directories: e.g., \texttt{/bin/}, \texttt{oasis-multisubject-sample/}.
	&&& Use the trailing slash for clarity in all cases.
	&& References to named variables: e.g., \texttt{\$(SUBJECTS)}, \texttt{\$<}, \texttt{\%}, \texttt{\$PWD}.
	&& Command-line flags written in-line: e.g., \texttt{-n}, \texttt{-lart}.
	& Don't monospace:
	&& \textbf{Don't double up on quotes and monospacing.} If both would be used, prefer quotes. This is to avoid a complete monospacing overload.
	&& Abstractions or filetypes: e.g., ``Makefiles are usually located \ldots'' or ``The nifti files \ldots''
	&& Programs typically used through a GUI: e.g. OpenOffice, gimp.
	&& Tool suites, like FSL or AFNI.
	&& The names of operating systems.
	&& References to the grid engine in any form.
	&&& Use the article with the abbreviation: ``the SGE.''
	&&& ``Grid engine,'' not ``gridengine.'' This is Oracle/SG's usage, although they capitalize it, which we won't do.
	&& Subject identifiers.
	& Oxford comma = yes: e.g. ``X, Y\textcolor{red}{,} and Z,'' not ``X, Y and Z.''
	& Don't use em dashes (--) for emphasis. They should appear in pairs the majority of the time.
	& Margin notes and footnotes:
	&& Use margin notes to call out things a new reader would find useful:
	&&& Define important terms with a [?].
	&&& Call out important notes with a [!].
	&& Use footnotes for things that could be completely ignored: advanced features, humorous commentary, etc\ldots.
	& \LaTeX{} swallows up two dashes (\dd). Use \verb!\dd! to get a double dash. You can use a space after this command and \LaTeX will ignore it.
	& \textbf{Skull-strip:} Two words, always hyphenated.
	& Headers:
	&& Chapters and Sections Are in Title Case
	&& Subsections in sentence case
	&& Figure captions in sentence case
	& Numerals:
	&& Spell out numbers less than 10, except the ones \LaTeX{} generates.
	&& Always spell out a number at the beginning of a sentence. However, it usually better to recast the sentence to avoid this.
	&& If you must refer to multiple number things, try:
	&&& Using A, B, C \ldots{} instead: ``Process A will overwrite foo.out just in time for process B to use it.''
	& Plurals
	&& Since we may have to pluralize some weird things: No apostrophe between the stem and the ``s,'' no matter what. Use the same formatting (e.g. monospacing). If possible, recast the sentence.
	& M/makefile
	&& ``a makefile'' is an abstract noun referring to any make script (cf. ``a shell script''). This should be capitalized when appropriate.
	&& ``the Makefile'' is the top-level makefile in a directory, usually named \texttt{Makefile} by IBIC conventions. This reference to the file with the same name does not need to be monospaced.
	& Backticks and single quotes in monospaced font:
	&& Use \texttt{\textbackslash\`{}\{\}} to get the correct backtick symbol.
	&& There is currently no support for a straight single quote in the defaulty \verb!\ttfamily! font or Inconsolata at present.
	& Ellipses
	&& Spaces before and after: ``this \ldots{} and that.''
	&& Use \verb!\ldots!
	& FreeSurfer
\end{easylist}

